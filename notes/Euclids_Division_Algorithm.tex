\title{Euclid's Division Algorithm }
\date{\today}

\documentclass[12pt]{article}
\usepackage{amsmath, amssymb,amscd}
\usepackage{enumitem}
\usepackage[dvipsnames]{xcolor}

\usepackage{listings}
\lstloadlanguages{Ruby}
\lstset{%
  basicstyle=\ttfamily\color{black},
  commentstyle = \ttfamily\color{red},
  keywordstyle=\ttfamily\color{blue},
  stringstyle=\color{orange}}


\begin{document}
\maketitle


Let's say you get hold of someone's Driver's Licence number and would like to figure out their date of birth. We can use a little bit of number theory to do that.

\paragraph{Euclid's Division Algorithm}
If $a$ and $b$ are integers then $a,b$ can be expressed as $a = bq + r$  where $ 0 \leq  r < b$
Moreover these integers $q$ and $r$ are unique because if we assume there are two other integers $q1,r1$ 
such that   $ 0 \leq  r, r1 < b$ 


  \begin{align}
 a = bq + r = bq1 + r1 \\
 b(q-q1) = r -r1  => b | r -r1 \\
  \end{align}
 since $r,r1 < b$ , $b|r-r1$ is a cotradiction

\paragraph{Euclidean Algorithm to find $gcd(34,126)$ }
  \begin{align}
    126 = 3 \cdot 34 + 24 \\
    34 = 1 \cdot 24 + 10 \\
    24 = 2 \cdot 10 + 4 \\
    10 = 2 \cdot 4 + 2 \\
    4 = 2 \cdot 2 + 0\\
    gcd(34,126) = 4
  \end{align}

This is the ruby script I use 
\begin{lstlisting}[language=Ruby]
def gcd(a,b)
  (t= a ; a =b ; b =t) if a < b 
  return b if a % b ==  0 
  gcd(b, a%b) 
end
\end{lstlisting}

\paragraph{gcd as as linear combination of numbers}
  If, $d = gcd(a,b)$, then $d$ can be written as $d = ma + kb$. From above,
  \begin{align}
    24 = 126 - 3 \cdot 34 \\
    10 = 34 - 1 \cdot 24 \\
    4 = 24 - 2 \cdot 10 \\
    2 = 10 - 2 \cdot 4
  \end{align}
  Substituting back,
  \begin{align}
    2 = ( 34 -24 ) - 2(24 - 2 \cdot 10) \\
    2 = 34 - 126 + 3 \cdot 34  -2 ( 126 - 3 \cdot 34 - 2( 34 -24)) \\
    2  = 4 \cdot 34  - 126  - 2\cdot126 + 6\cdot34 + 4\cdot34 - 4\cdot24 \\
    2 = 14\cdot34 - 3\cdot126 - 4\cdot 126 + 12\cdot34 \\
    2 = 26\cdot34 - 7\cdot126
  \end{align}



\end{document}
