\documentclass[12pt]{article}
\usepackage{amsmath, amssymb,amscd}
\usepackage{xypic}
\usepackage{setspace}
\usepackage{enumerate}
\pagestyle{empty}
\setlength{\parindent}{0in}
\oddsidemargin 0in
\textwidth 6.25in
\topmargin 0in
\textheight 9in
\doublespace
\usepackage{amsthm}
\usepackage{enumitem}
\usepackage{tikz}
\usepackage{fixltx2e}
\usetikzlibrary{shapes.geometric}
\usetikzlibrary{arrows,decorations.markings}

\newenvironment{faq}{\begin{description}[style=nextline]}{\end{description}}
\tikzset{
    polygon n/.code=\gdef\polygonN{#1}, % Save N in a global macro
    polygon/.style={
        regular polygon,
        regular polygon sides=#1,
        polygon n=#1,
        draw,
        minimum size=2cm,
        outer sep=0pt
    },
    mirror polygon/.style={
        insert path={
            let \p1 = (A.corner #1),
                \p2 = (A.side #1) in (\p1)
            ($ 2*($(\p1)!(A.corner 1)!(\p2)$) - (A.corner 1) $) % Path needs to be started before the foreach
            \foreach \n in {2,...,\polygonN} {
                -- ($ 2*($(\p1)!(A.corner \n)!(\p2)$) - (A.corner \n) $)
            } -- cycle
        }       
    }
}
\begin{document} 

\begin{document}

\begin{faq}
\item[Cayley table for D\textsubscript3]
A symmetry can be described as an operation when performed on a object results in the same object.\\
Let's explore symmetry of D\textsubscript3, an equilateral triangle

\begin{tikzpicture}
  \node (A) [draw,regular polygon, regular polygon sides=3, minimum size=2cm,outer sep=0pt] {};
  \foreach \n in {1,...,3} {
    \node at (\n) [anchor=360/3*(\n-1)+270] {\scriptsize\texttt{ \n}};
  }
\end{tikzpicture}



\end{faq}
\end{document}
